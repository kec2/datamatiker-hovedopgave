\paragraph{M�l og planer for uge 6}

Vi skal have diskuteret vore mappe struktur og placeringen af vores filer. Der er en reel risiko for at miste overbliket over alle vores noter.
Den nye prototype skal udvikles.
Der skal afholdes en session med Kim, Dion og Claus, hvor vi f�r fastlagt hvilket felter og informationer der skal v�re i indtastningsdelen.
Vi skal starte p� at indsamle informationer om hvordan pr�sentationsdelen/beregningsdelen udformes.
Ide: For at indsamle krav til pr�sentationsdelen af applikationen kunne der afholdes en brainstorming session / m�de med Dion og Kim. Vi skal sk�re ind til benet og f� de vigtige ting p� banen. Hvad skal en pr�sentation for kunden indeholde. Hvilke grafer og p� hvilke tal skal graferne laves.
Vi skal kigge n�rmere p� identificere den h�rdeste n�d - dette skal vi komme n�rmere ind p�.
Vi skal kigge p� et udkast/vores tanker om problemformuleringen.
Vi skal unders�ge om det er planen at der bliver rene indtastnings opgaver for sekret�rerne.
Vi skal n�rmere et studieomr�de i forbindelse med rapporten.
Vi skal (indtil videre) have fastlagt de principper vi arbejder efter, hvilke ligger grund for vores sytemudviklingsmetode.
Kontorstol og telefon eftersp�rges.


Den h�rde n�d
P� dette stadie er det sv�rt at definere et enkelt omr�de som den h�rdeste n�d, da vi er i en proto-typing fase. I denne fase er det essentielle at komme frem til 'de rigtige' krav, ud fra det input vi f�r fra virksomheden, hvorfor man kan sige at kravspecifiseringen er den h�rde n�d.

Den umiddelbare mest risiko-fyldte n�ste ting vi skal igang med er tilgangen til databasen. Dette er risiko-fyldt i forhold til om det kan komme til at virke p� EIKs systemer hos SDC, ligesom at det er risiko-fyldt i forhold til om vi konstruere tilgangen rent teknisk set. 

Overordnet for projektet er der andre risiko-omr�der, som alle er kode-tekniske problemstillinger. Nogle af disse er fremtidige og kan opsummeres som:

Udprintningsmodulet
Pr�sentationsdelen, herunder grafisk visualisering i form af grafer etc.
Fremskrivningsdelen