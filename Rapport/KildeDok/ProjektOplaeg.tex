PRIVATE BANKING V�RKT�J\\

Indledning: Private banking best�r i r�dgivning overfor formuende kunder indenfor investeringspleje, pensions- og skatteforhold, f.eks. i forbindelse med til- og fraflytning udlandet.\\Grundlag: Foruds�tningen for en professionel r�dgivning af kunder indenfor Private Banking er et overblik over kundens - og �gtef�lles indt�gtforhold, formueforhold, arbejdssituation, pensionsopsparing, forsikringsd�kninger samt fremtidige forventninger.\\ Dette overblik skabes i dag via registrering af alle ovenn�vnte oplysninger i Excel, hvorefter overblikket generes. Herefter skal man indtaste fornyede oplysninger for at simulere andre situationer i forbindelse med r�dgivningen af kunden. Dette kan p.t. v�re en besv�rlig proces.\\

Fremtidigt v�rkt�j:
Projektgruppen opgave vil v�re at finde frem til en mere automatiseret l�sning, som kan erstatte det nuv�rende excel ark. \\ Vi kan forestille os f�lgende l�sningsmulighed:\\ En database, hvor kundens oplysningen registreres i forskellige tabeller. \\ En r�kke andre tabeller/databaser med nyttige oplysninger, f.eks. \\

\begin{itemize}
\item offentlige pensioner 
\item test vedr. risikoanalyse vedr. risikoprofil
\item pensions- og d�kningsforhold for de foretrukne samarbejdspartnere/forsikringsselskaber
\item v�rdipapirer over foretrukne portef�ljer, f.eks. Eik Banks modelportef�lje   
\item skatteforhold i foretrukne udlande, f.eks. Frankring, Spanien, England m.fl.
\item g�ngse realkreditl�n, beregningsmotor, oml�gning af l�n
\end{itemize}

Noget af ovenn�vnte kunne t�nkes at v�re links til andre websites.\\

Her ovenp� l�gges et brugerinterface, hvor resultatet af kundens oplysninger vises i pr�sentabel format ( mulighed for udskrift). Det skal endvidere v�re muligt at simulere forskellige muligheder/situationer under m�det med kunden, b�de med hensyn til indt�gtsforhold, pension- og d�kningsforhold, Investeringspleje samt skattem�ssige forhold, dels p� grundlag af kundens egne oplysninger og dels de faste oplysninger, jf. ovenfor. Kunden vil dermed f� et relevant og grundigt overblik/grundlag for at tr�ffe beslutninger.