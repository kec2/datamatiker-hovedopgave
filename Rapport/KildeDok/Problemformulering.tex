\newpage
\section{Problemformulering}

Ud fra de betragtninger vi har gjort om det eksisterende system, hvor EIK Bank anvender Excel regneark til simulering af private kunders �konomi, ses det tydeligt at et nyt system vil kunne bidrage med v�sentlige forbedringer. Disse forbedringer vil til dels komme r�dgiveren og dennes arbejdsgang til gode og dels vil de komme kunden til gode ved et hurtigere og bedre overblik, i form af et mere professionelt udprint. Systemet skal ikke lave v�sentligt om p� de processer der ligger i den nuv�rende arbejdsgang, men g�re det lettere at udf�re arbejdet. For at vi kan finde ud af hvordan det nye system skal konstrueres, er det n�dvendigt at analysere en r�kke forhold, hvoraf problemformuleringen kan defineres:

%Hvordan kan vi udvikle et bruger- og udvidelsesvenligt system til private banking, som giver kunden et forbedret overblik over deres �konomiske forhold, i forbindelse med registrering af kunders privat�konomi.\\
%Vi vil yderligere unders�ge hvordan vi kan skabe et godt design af brugergr�nsefladen og om vi med fordel kan inddrage elementer fra eksperimentiel systemudvikling i processen. Vi er endvidere interesserede i at implementere systemet ved hj�lp af design patterns, hvorfor vi vil unders�ge hvilke der kan v�re relevante for vores system.

EIK Bank har et krav om at systemet skal v�re brugervenligt, kunne udskrive professionelt udseende rapporter til kunden og skabe gennemskuelighed og overskuelighed af kundens �konomi. Hvordan kan vi opfylde dette krav ved inddragelse af eksperimentielle systemudviklingsmetoder og teori om brugervenlighed?

Et andet krav siger at systemet skal v�re modul�rt og udvidelsesvenligt. Kan vi ved brug af design patterns opfylde dette krav, og i s� fald med hvilke design patterns?



%\begin{itemize}
%\item Hvordan f�r vi skabt et optimalt design af brugergr�nsefladen?

%\item Vi er interesserede i at implementere systemet ved hj�lp at design patterns. Hvilke fordele kan vil f� for systemet? Hvilke design patterns er relevante for denne applikation?

%\item Vi er endvidere interesserede i at inddrage aspekter af eksperimentiel systemudvikling i projektet. Hvordan vil dette p�virke projektforl�bet?

%\item Hvad kan vi g�re for at systemet ikke tilpasses den ene r�dgivers arbejdsform mere end den andens?

%\item Hvordan kan vi forbedre systemets visualiseringsgrad overfor kunden, med hensyn til udprint af data?
%\end{itemize}






