\paragraph{I dag d. 23-01-06}

\begin{itemize}
\item Skal have overblik over rapporten. Hvordan er vores forestilling om den overordnede struktur. Der skal v�re fyld til 100 sider (puha :). Vi skal i gang med at skrive p� rapporten med det samme og skal absolut ikke vente til sidst med at g� i gang.
\item Vi har snakket om at vi skal kunne tage udgangspunkt i de forskellige omr�der n�r vi diskuterer, eller rettere fors�ge at adskille de forskellige omr�der: analyse, design, implementering. Fordelen er f.eks at vi ikke begr�nser vores kreative tankegang af sn�vre kodetankegange i en analyse situation hvor kreative muligheder skal udforskes. Alts� skal vi fors�ge at v�re mere bevidste p� hvilken situation vi befinder os i og tage den rette kasket p�. Alts� ikke noget med at tage analysekasket OG implementeringskasket p�. OFTE FORUDS�TTER DE FORSKELLIGE SITUATIONER HINANDEN OG B�R DERFOR UDF�RES I GENSIDIGT SAMSPIL ;)
\item Vi opfatter hver is�r de svar vi f�r forskelligt, hvilket kan ses som et resultat af d�rlig m�de-teknik og/eller som et resultat af et kommunikationsvanskeligheder (i gruppen og mellem EIK og os).
\item Vores m�de i dag var alt for uprofessionelt, vi skal foreberede os bedre (roller, ikke snakke i munden p� hinanden, have sp�rgsm�l klar f�r m�det, f�lles agenda). 

\item Vi har allerede nu en klar forestilling om at vi skal have en database ind over vores l�sning. Efter at have taget en lille samtale om fordele og ulemper med database eller ej.
\item EIK har en filserver, som databasen kan k�re p�. Dette skal testes.
\item Vi har ogs� l�st snakket om hvilke SU-metoder der skal indover udviklingsforl�bet. Der er stemning for eksperimentel prototyping og brug af UML.
\end{itemize}
