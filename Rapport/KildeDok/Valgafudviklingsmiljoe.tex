\section{Vores valg af udviklingsmilj�}

Vi har i gruppen hver is�r forskellige programmeringssprog som vi br�nder for og har arbejdet mest med. Nogle er til Java, nogle er til C\# og andre er til noget helt tredje. Dog havde vi den f�llesn�vner at alle i gruppen havde det samme valgfag p� 4. semester, nemlig .Net \& Visual Studio 2003. Her dannede vi ogs� gruppe, for at lave valgfagsprojekt i samme fag. I valgfagsprojektet skulle vi lave en applikation i Visual Studio 2003 og her valgte vi C\# som programmeringssprog. Vi skulle endvidere bygge en Access database og fik dermed �velse i at koble de to v�rkt�jer sammen.

Da vi begyndte p� hovedopgaven var .Net frameworket lige udkommet i version 2.0 og Visual Studio var udgivet i den nye version 2005. EIK Bank har ikke haft nogen pr�ferencer over for  hvilket sprog applikationen udvikles i, da de ikke selv har udviklere siddende i huset. Eneste anm�rkning fra bankens side var at vi godt m�tte tage videre\-udviklings\-muligheder med i overvejelserne. Derfor lod vi vores f�lles erfaring fra 4. semester v�re et af argumenterne for at v�lge .Net platformen og sproget C\#. Et andet argument var at vi gerne ville arbejde mere med .Net og C\# og l�re platformen bedre at kende, da den er ``oppe i tiden''. Derfor mener vi ogs� at vores valg er med til at ``fremtidssikre'' systemet, da banken engang i fremtiden med sikkerhed kan finde ekstern arbejdskraft til videreudvikling af systemet i og med at platformen er udbredt og anvendes mange steder.
