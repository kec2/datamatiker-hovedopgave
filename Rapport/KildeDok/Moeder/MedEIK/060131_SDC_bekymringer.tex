\paragraph{M�de med Klaus Pommern og Johann, Bekymringer omkring SDC, d. 31-01-06}

\textbf{Vi fandt ud af f�lgende p� m�det.}
Indtastning / justering skal v�re mulig hos kunden. Dette pga af at der ofte vil v�re rettelser og tilf�jelser ude hos kunden. Det vil ogs� gavne fleksibiliteten i systemet. Der skal selvf�lgelig ogs� v�re mulighed for at lave justeringer i kundernes �konomi i henhold til deres �nsker.
Dette kr�ver nok at vi laver en lille 'buffer-DB'  p� applikationssiden, hvor data opbevares indtil der kan synkroniseres med 'moder-databasen'.
Vi fik sl�et fast at det ikke er en mulighed at f� SDC til at s�tte en server op p� dette stadie i projektet. Det er alt alt for dyrt.
Aspektet backup skal medtages i overvejelser omkring data opbevaring.
Fremtidsmuligheder:
Systemet skal naturligvis v�re modulopbygget s�ledes at der eksisterer en mulighed for at have 2vejs kommunikation med SDC database, evt Lotus notes.
Vi m� i den nuv�rende situation g� ud fra at vi skal lave en l�sning med en Access database p� EIK�s f�llesdrev.

\textbf{HUSK: }
Hold os for �je at vi skal 'crack hardest nut first' = chNf.

\textbf{En vigtig reflektion p� baggrund af vores lille m�de:}
Vi skal passe p� med at love for meget. Ogs� selvom vi m�ske ikke direkte lover noget, men derimod erkl�rer interesse for aspektet, skal vi passe meget p�. Der er meget kort vej fra at erkl�re interesse til at love. Misforst�elsen opst�r relativt nemt og det er noget vi skal v�re obs p�. Yderligere skal vi nok lige tage teten op p� fredags briefingen og forklare vores tanker omkring dette.
