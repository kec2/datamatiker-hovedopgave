\paragraph{Morgenm�de 24-01-06}

\begin{itemize}
\item Karakteristik af EIK: Ikke EDB-kyndige medarbejdere, brugere p� normalt nivau.

\item Valg af database?
Access indtil videre. Da der ikke bliver meget pres p� med 4 medarbejdere. Det ligger ogs� i kortetene at vi b�r g� efter en kostfri l�sning. Afpr�v en simpel database p� EIK filserveren.

\item Webservice?
Kan filserveren bruges til webservices

\item Rapportstruktur:
Kig gamle rapporter igennem.

\item Latex rapport header:
Lav en header til texfilerne til rapporten.

\item Videoproto af private banking session:
F� Dion til at vise hvordan et realistisk kunderforl�b ser ud. Se nuv�rende arbejdsgange og f� ham til at s�tte ord p� de nuv�rende problemer.

\item Lave en tidsoversigt:
Lave en tom 'skal' over tidsforl�bet.

\item Login delen:
P�begyndelse af logindelen er en mulighed allerede nu.

\item Kigge p� mulige faldgrupper i forl�bet som det ser ud nu (mild risikoanalyse).

\item Fastl�gge filosofierne:
\begin{itemize}
\item Crack the hardest nut first.
\item 'Stille dumme sp�rgsm�l filosofien' Ingen sp�rgsm�l er for dumme. Modvirker kommunikationsvanskeligheder.
\item KISS-Keep It Simple Stupid / KISBI: Keep It Simple But Intelligent
\end{itemize}

\item Fastl�gge arbejdsformerne:
Spiralmodellen.
Daglige m�der efter Scrum-forbillede.
Der er 7 iterationer. 1 uge er planl�gning og installation. Sidste 2 uger er til rapportskrivning.
K�re efter fastlagte ugentlige m�l. Det kan v�re et delsystem eller en papirmodel. el. M�let skal v�re f�rdigt n�r vi g�r hjem torsdag.
Afholde et ugentligt informationsm�de med EIK kontakterne. Prim�rt for at vise fremskridtet, engagement og for at hindre at vi kommer for langt ud p� et sidespor. Dette ugentlige m�de afholdes fredag formiddag (11-11.30 tiden).
NOTE: Vi skal informere EIK medarbejdere p� forh�nd om hvor meget tid vi regner med at der g�r ved samtaler og andre sessioner.
\end{itemize}

